% !Mode:: "TeX:UTF-8"
% !TEX program  = xelatex

% \documentclass{cumcmthesis}
\documentclass[withoutpreface,bwprint]{cumcmthesis} %去掉封面与编号页,电子版提交的时候使用。


\usepackage[framemethod=TikZ]{mdframed}
\usepackage{url}   % 网页链接
% \usepackage{subcaption} % 子标题
\title{基于宽度优先搜索的穿越沙漠最优路径规划}

\begin{document}

 \maketitle
 \begin{abstract}


\keywords{宽度优先搜索\quad  图搜索\quad   最优路径规划\quad  规划模型}
\end{abstract}

%目录  2019 明确不要目录,我觉得这个规定太好了
%\tableofcontents

%\newpage

\section{问题重述}

穿越沙漠问题是一个在约束条件下求最优解的问题,具体来说,指在给定的地图上,玩家在游戏开始时获得一定的初始资金,可以在起点购买一定数量的水和食物,然后从起点出发,穿过沙漠抵达终点。在穿越沙漠中,玩家所携带的水和食物不能呢超越其最大背负重量,但必须大于等于当天的消耗,从而不至于渴死或者饿死。穿越途中会遇到不同的天气,也可以在矿山获得资金,或者在村庄购买资源,抵达终点之后也可以退回多余的水和食物。游戏要求在规定的时间到达终点,在终点拥有的资金则多多益善。

我们需要建立数学模型解决如下具体问题:

\subsection{问题一}
只有一名玩家,游戏开始时,游戏时间内每天天气情况就全部已知,在此条件下求解一个最优策略,使得到达终点时持有最多的资金。

\section{问题分析}
沙漠的地图可以抽象为一张无向连通图,我们可以通过宽度优先搜索(BFS)来
\section{问题一的分析}
问题一中游戏全程的天气情况都是已知的,

\section{基本假设}

\section{变量说明}

\begin{table}[!htbp]
    \caption{符号说明}\label{tab:001} \centering
    \begin{tabular}{ccc}
        \toprule[1.5pt]
        符号 & 意义 & 单位\\
        \midrule[1pt]
        $d$ & 当前日期 & \\
        $p$ & 所在的区域编号 & \\
        $f$ & 所带的食物 & 箱\\
        $w$ & 所带的水 & 箱\\
        $m$ & 所带的资金 & 元\\
        \bottomrule[1.5pt]
    \end{tabular}
\end{table}

\section{模型的建立与求解}


\section{参考文献与引用}

%参考文献
\begin{thebibliography}{9}%宽度9
    % \bibitem[1]{liuhaiyang2013latex}
    % 刘海洋.
    % \newblock \LaTeX {}入门\allowbreak[J].
    % \newblock 电子工业出版社, 北京, 2013.
    \bibitem{mathematical-modeling}
    全国大学生数学建模竞赛论文格式规范 (2020 年 8 月 25 日修改).
    \bibitem{latexstudio} \url{https://www.latexstudio.net}
\end{thebibliography}

\newpage
%附录
\begin{appendices}

\section{程序}

\begin{lstlisting}[language=matlab]

\end{lstlisting}

\end{appendices}

\end{document} 